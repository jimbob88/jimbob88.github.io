\documentclass{article}[12pt]
\usepackage[english]{babel}
\usepackage{amsmath}
\usepackage{amssymb}
%\usepackage{pgfplots, capt-of}
%\pgfplotsset{width=10cm,compat=1.9}
\title{Invariant Points and Lines}
\author{James Blackburn}
\date{January 2022}
\begin{document}
\maketitle
\section{Definitions}
\begin{description}
\item[Matrix Transformation] A linear transformation represented by a 2x2 matrix.
\item[Invariant Point] A point (coordinate) that is not changed by a matrix transformation.
\item[Invariant Line] A line (in the form $y=mx+c$), which is not effected by a matrix transformation.
\item[Line of invariant points] A line of points which are individually all invariant points.
\end{description}

\section{Calculating matrix transformations}
Matrix transformations are extremely simple and can be done in the following way:
\begin{enumerate}
\item Define the transformation matrix i.e.:
 \(\boldsymbol{T} = \begin{bmatrix} 0 & -1 \\ -1 & 0\end{bmatrix}\) (reflection in the line $y=-x$)
\item Multiply this by a point i.e.: \(\vec{v_1} = \begin{bmatrix} 2 \\ -1\end{bmatrix}\)
\item Multiply the two values \(\boldsymbol{T} \times \vec{v_1} = \begin{bmatrix} 0 & -1 \\ -1 & 0\end{bmatrix}\begin{bmatrix} 2 \\ -1\end{bmatrix} = \begin{bmatrix} 1 \\ -2\end{bmatrix}\)
\end{enumerate}

\section{Finding invariant points}
First of all, the origin $\begin{bmatrix}0\\0\end{bmatrix}$ is never effected by a linear transformation. Why?
\begin{minipage}{\textwidth}
\begin{enumerate}
\item Define the transformation:
\(\boldsymbol{T} = \begin{bmatrix} a & b \\ c & d\end{bmatrix}\)
\item Define the vertex $ \begin{bmatrix} 0 \\ 0\end{bmatrix}$
\item Multiply the matrix by the vertex: \(\begin{bmatrix} a & b \\ c & d\end{bmatrix}\begin{bmatrix} 0 \\ 0\end{bmatrix} = \begin{bmatrix} a \times 0 + b \times 0 \\ c \times 0 + d \times 0\end{bmatrix} = \begin{bmatrix} 0 \\ 0\end{bmatrix}\)
\end{enumerate}
\end{minipage}
\\
To find a line of invariant poins from the matrix $\begin{bmatrix} 0 & -1 \\ -1 & 0\end{bmatrix}$ (reflection in the line $y=-x$)
\begin{enumerate}
    \item Define the equation: $\begin{bmatrix} 0 & -1 \\ -1 & 0\end{bmatrix}\begin{bmatrix} x \\ y\end{bmatrix}=\begin{bmatrix} x \\ y\end{bmatrix}$
    \item Multiply it out to give: $\begin{bmatrix} -y \\ -x\end{bmatrix}=\begin{bmatrix} x \\ y\end{bmatrix}$
    \item Then expand that to the equations: $-y = x$ and $-x = y$
    \item Rearrange this to give: $y = -x$ and $y = -x$.
    \item Clearly these are both the same, therefore, all the points are not effected by the reflection in the line $y=-x$ (this makes logical sense)
\end{enumerate}

\begin{minipage}{\textwidth}
\section{Finding invariant lines}
Invariant lines are different to "lines of invaraint points" because the individual points on the line can move, but the line itself remains the same!

The best way to work this out (in my opinion):

\begin{enumerate}
    \item Define the matrix transformation in the variable $\boldsymbol{T}$: 
    
    \(\boldsymbol{T} = \begin{bmatrix} -1 & 1 \\ -4 & 3 \end{bmatrix}\)
    \item Multiply the transformation by the general formula for lines $y = mx + c$ and $y = mx^{'}+c$.
    \item 
    
    \(\begin{bmatrix} -1 & 1 \\ -4 & 3 \end{bmatrix}\begin{bmatrix} x \\ mx+c \end{bmatrix} = \begin{bmatrix} x^{'} \\ mx^{'}+c \end{bmatrix}\)
    \item 
    \(-x + mx+c = x^{'} \\
    -4x + 3(mx+c) = mx^{'} + c\)
    \item Substitute the value $x^{'}$ into the second formula!
    \begin{equation}
        \begin{aligned}
        -4x + 3(mx+c) &= m(-x + mx+c) + c \\
        -4x + 3mx + 3c &= -mx + m^{2}x+mc + c \\
        2c - mc &= -4mx + m^{2}x + 4x \\
        c(2-m) &= x(-4m + m^{2} + 4) \\ 
        c(2-m) &= x(m^{2} - 4m + 4) \\
        c(2-m) &= x(m-2)^2 \\
        x(m-2)^2 - c(2-m) &= 0 \\
        x(m-2)^2 + c(m-2) &= 0 
        \end{aligned}
    \end{equation}
    \item You can see, to solve this formula for $m$, $m = 2$
    \item Rewrite the formula with the value $m = 2$
    \item Therefore, we can write the following formula: $y = 2x + c$
    \item Where $c$ is any value!
    \item Therefore there is an invariant line for all lines with the gradient $2$
\end{enumerate}
\end{minipage}
% First of all, we can define the 2x2 matrix inversion in the variable $\boldsymbol{R}$ as:
% \begin{gather} 
% \text{Where }\theta \text{ is the anti-clockwise angle of rotation.} \\
% \boldsymbol{R}  = \begin{bmatrix}\cos \theta & -\sin \theta \\ \sin \theta & \cos \theta \end{bmatrix} 
% \end{gather} 

\end{document}