% Options for packages loaded elsewhere
\PassOptionsToPackage{unicode}{hyperref}
\PassOptionsToPackage{hyphens}{url}
%
\documentclass[
]{article}
\usepackage{amsmath,amssymb}
\usepackage{lmodern}
\usepackage{iftex}
\ifPDFTeX
  \usepackage[T1]{fontenc}
  \usepackage[utf8]{inputenc}
  \usepackage{textcomp} % provide euro and other symbols
\else % if luatex or xetex
  \usepackage{unicode-math}
  \defaultfontfeatures{Scale=MatchLowercase}
  \defaultfontfeatures[\rmfamily]{Ligatures=TeX,Scale=1}
\fi
% Use upquote if available, for straight quotes in verbatim environments
\IfFileExists{upquote.sty}{\usepackage{upquote}}{}
\IfFileExists{microtype.sty}{% use microtype if available
  \usepackage[]{microtype}
  \UseMicrotypeSet[protrusion]{basicmath} % disable protrusion for tt fonts
}{}
\makeatletter
\@ifundefined{KOMAClassName}{% if non-KOMA class
  \IfFileExists{parskip.sty}{%
    \usepackage{parskip}
  }{% else
    \setlength{\parindent}{0pt}
    \setlength{\parskip}{6pt plus 2pt minus 1pt}}
}{% if KOMA class
  \KOMAoptions{parskip=half}}
\makeatother
\usepackage{xcolor}
\IfFileExists{xurl.sty}{\usepackage{xurl}}{} % add URL line breaks if available
\IfFileExists{bookmark.sty}{\usepackage{bookmark}}{\usepackage{hyperref}}
\hypersetup{
  hidelinks,
  pdfcreator={LaTeX via pandoc}}
\urlstyle{same} % disable monospaced font for URLs
\usepackage{longtable,booktabs,array}
\usepackage{calc} % for calculating minipage widths
% Correct order of tables after \paragraph or \subparagraph
\usepackage{etoolbox}
\makeatletter
\patchcmd\longtable{\par}{\if@noskipsec\mbox{}\fi\par}{}{}
\makeatother
% Allow footnotes in longtable head/foot
\IfFileExists{footnotehyper.sty}{\usepackage{footnotehyper}}{\usepackage{footnote}}
\makesavenoteenv{longtable}
\setlength{\emergencystretch}{3em} % prevent overfull lines
\providecommand{\tightlist}{%
  \setlength{\itemsep}{0pt}\setlength{\parskip}{0pt}}
\setcounter{secnumdepth}{-\maxdimen} % remove section numbering
\ifLuaTeX
  \usepackage{selnolig}  % disable illegal ligatures
\fi

\author{}
\date{}

\begin{document}

\hypertarget{sampling-techniques}{%
\section{Sampling Techniques}\label{sampling-techniques}}

\hypertarget{types-of-sampling-technique}{%
\subsubsection{Types of sampling
technique}\label{types-of-sampling-technique}}

\hypertarget{random}{%
\paragraph{Random}\label{random}}

\begin{itemize}
\item
  Systematic Sampling
\item
  Simple Random Sampling
\item
  Stratified Sampling
\end{itemize}

\hypertarget{non-random}{%
\paragraph{Non-random}\label{non-random}}

\begin{itemize}
\item
  Quota Sampling
\item
  Opportunity Sampling
\end{itemize}

\hypertarget{definitions-examples}{%
\subsubsection{Definitions \& Examples}\label{definitions-examples}}

\hypertarget{systematic-sampling}{%
\paragraph{Systematic Sampling}\label{systematic-sampling}}

Systematic sampling is where you start at a random position in a list,
and then take every \(n^{th} \text{ term}\) and create a sample of the
population this way.

For example:

\begin{gather*}
  \text{The full population is 20 people, the following list shows them:}
  \\
  [1, 2, 3, 4, 5, 6, 7, 8, 9, 10, 11, 12, 13, 14, 15, 16, 17, 18, 19, 20]
  \\
  \text{Pick a random number to start on:}
  \\
  3
  \\
  \text{Select every 5th number after this:}
  [3, 8, 13, 18]
\end{gather*}

Note: The sampling frame is usually randomised to decrease bias
(grouping people alphabetically could cause issues where certain members
of the population are over or under-represented)

\hypertarget{simple-random-sampling}{%
\paragraph{Simple Random Sampling}\label{simple-random-sampling}}

Given a list of numbers, you just randomly select a sample of them.
I.E., generate 10 random numbers on a calculator, and then pick these
people.

\begin{minipage}{\textwidth}

\hypertarget{stratified-sampling}{%
\paragraph{Stratified Sampling}\label{stratified-sampling}}

Stratified Sampling is where you take the population and split them into
different groups. You then take these groups and proportionally sample
based off their numbers.


For example:

\begin{gather*}
  \text{The full population is 100 people. There are 60 women, and 40 men.}
  \\\therefore 
  \text{60\% women and 40\% men}
  \\
  \text{You are then told that you must do a sample of 50 of these people}
  \\
  \therefore 60\% \times 50 = 30, 40\% \times 50 = 20
  \\
  \text{You then take a simple random sample from these numbers}
\end{gather*}
\end{minipage}
\hypertarget{quota-sampling}{%
\paragraph{Quota Sampling}\label{quota-sampling}}

Quota Sampling is where the population is divided into groups of
characteristics (specifically selected by the researcher). For example,
before a researcher starts, they may decide that they want to interview
\(30\%\) women and \(70\%\) men. The researcher then interviews people
and assesses their group, this continues until each quota has been
filled. If someone refuses to be interviewed, one can just move on!

\hypertarget{opportunity-sampling}{%
\paragraph{Opportunity Sampling}\label{opportunity-sampling}}

A good example of this is when someone waits outside a shop and picks a
random person to interview whenever someone steps outside the door, it
doesn't matter if the person refuses to be interviewed, because you can
just wait for the next person!

\hypertarget{advantages-disadvantages}{%
\paragraph{Advantages \& Disadvantages}\label{advantages-disadvantages}}

\hypertarget{systematic-sampling-1}{%
\paragraph{Systematic Sampling}\label{systematic-sampling-1}}

\begin{longtable}[]{@{}
  >{\raggedright\arraybackslash}p{(\columnwidth - 2\tabcolsep) * \real{0.46}}
  >{\raggedright\arraybackslash}p{(\columnwidth - 2\tabcolsep) * \real{0.54}}@{}}
\toprule
Advantages & Disadvantages \\
\midrule
\endhead
Simple and quick to use & A sampling frame is required \\
Suitable for large samples and large populations & It can introduce bias
if the sampling frame is not random \\
\bottomrule
\end{longtable}

\hypertarget{simple-random-sampling-1}{%
\paragraph{Simple Random Sampling}\label{simple-random-sampling-1}}

\begin{longtable}[]{@{}
  >{\raggedright\arraybackslash}p{(\columnwidth - 2\tabcolsep) * \real{0.53}}
  >{\raggedright\arraybackslash}p{(\columnwidth - 2\tabcolsep) * \real{0.47}}@{}}
\toprule
Advantages & Disadvantages \\
\midrule
\endhead
Free of bias & Not suitable when the population or sample size is too
large \\
Easy and cheap to implement for small populations and small samples & A
sampling frame is required \\
Each sampling unit has a known and equal chance of selection & \\
\bottomrule
\end{longtable}

\hypertarget{stratified-sampling-1}{%
\paragraph{Stratified Sampling}\label{stratified-sampling-1}}

\begin{longtable}[]{@{}
  >{\raggedright\arraybackslash}p{(\columnwidth - 2\tabcolsep) * \real{0.42}}
  >{\raggedright\arraybackslash}p{(\columnwidth - 2\tabcolsep) * \real{0.58}}@{}}
\toprule
Advantages & Disadvantages \\
\midrule
\endhead
Sample accurately emulates the population structure & The random
selection within each strata suffers from the same issues as Simple
Random Sampling \\
Guarantees proportional representation of groups within a population &
Population must be clearly classified into distinct groups \\
\bottomrule
\end{longtable}

\hypertarget{quota-sampling-1}{%
\paragraph{Quota Sampling}\label{quota-sampling-1}}

\begin{longtable}[]{@{}
  >{\raggedright\arraybackslash}p{(\columnwidth - 2\tabcolsep) * \real{0.50}}
  >{\raggedright\arraybackslash}p{(\columnwidth - 2\tabcolsep) * \real{0.50}}@{}}
\toprule
Advantages & Disadvantages \\
\midrule
\endhead
Small samples can still be representative of a whole population &
Non-random sampling can introduce bias \\
Simple, quick \& cheap & Groups might be inaccurate \\
Allows a researcher to make comparisons between the groups of a
population & Increasing the scope of a study drastically increases the
cost of the study \\
No sampling frame & Non-responses give no meaningful data \\
\bottomrule
\end{longtable}

\hypertarget{opportunity-sampling-1}{%
\paragraph{Opportunity Sampling}\label{opportunity-sampling-1}}

\begin{longtable}[]{@{}
  >{\raggedright\arraybackslash}p{(\columnwidth - 2\tabcolsep) * \real{0.50}}
  >{\raggedright\arraybackslash}p{(\columnwidth - 2\tabcolsep) * \real{0.50}}@{}}
\toprule
Advantages & Disadvantages \\
\midrule
\endhead
Easy to carry out & Dependent on the individual researcher \\
Cheap & Most likely won't be representative \\
\bottomrule
\end{longtable}

\end{document}
