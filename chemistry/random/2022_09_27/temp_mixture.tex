\documentclass{article}
\usepackage[utf8]{inputenc}
\usepackage{gensymb}
\usepackage{amsmath}

\title{Maths formula}
\author{James Blackburn}
\date{September 2022}

\begin{document}

\maketitle

\section{The Original Formula}

\[
t_f = \frac{(c_1 \times m_1 \times t_1) + (c_2 \times m_2 \times t_2)}{(c_1 \times m_1) + (c_2 \times m_2)}
\]

Where:

\begin{itemize}
\item \(t_f\) is the final temperature of the mixture (\(\degree C\))
\item \(c\) is the specific heat capacity of the substance (\(J/kg K\))
\item \(m\) is the mass of the substance (\(kg\))
\item \(t\) is the temperature of the substance (\(\degree C\))
\end{itemize}

You can use this term with as many substances as you like (simply add each term in the pattern shown).

\section{Simplifying for water}

Because the \(c\) is common to both substances (water has a specific heat capacity of \(4184 \; J/kg K\), we can simplify the above equation:

\[
\begin{aligned}
    t_f &= \frac{c \times m_1 \times t_1 + c \times m_2 \times t_2}{c \times m_1 + c \times m_2} \\
    &= \frac{c(m_1 \times t_1 + m_2 \times t_2)}{c(m_1 + m_2)} \\
    &= \frac{m_1 \times t_1 + m_2 \times t_2}{m_1 + m_2} \\
\end{aligned}
\]

In relation to the above statement "\(c\) is common to both substances", this is not strictly true, but it is a good enough approximation for us engineers!
\end{document}
